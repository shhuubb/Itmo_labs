  %%%%%%%%%%%%%%%%%%%%%%%%%%%%%%%%% LAB-5 %%%%%%%%%%%%%%%%%%%%%%%%%%%%%%%%%%
%>>>>>>>>>>>>>>>>>>>>>>>>>> ПЕРЕМЕННЫЕ >>>>>>>>>>>>>>>>>>>>>>>>>>>>>>>>>>>

%\newcommand{\year}{2021 г.}  % Год устанавливается автоматически
\newcommand{\city}{Санкт-Петербург}  %  Футер, нижний колонтитул на титульном листе
\newcommand{\university}{Национальный исследовательский университет ИТМО}  % первая строка
\newcommand{\department}{Факультет программной инженерии и компьютерной техники}  % Вторая строка
\newcommand{\major}{Направление программная инженерия}

\newcommand{\education}{Образовательная программа системное и прикладное программное обеспечение}  % четвертая строка

%<<<<< Информация о кафедре

%>>>>> Назание работы
\newcommand{\reporttype}{ОТЧЕТ ПО ЛАБОРАТОРНОЙ РАБОТЕ} % тип работы, (главный заголовок титульного листа)
\newcommand{\lab}{Лабораторная работа}          % вид работы
\newcommand{\labnumber}{№ 3}                    % порядковый номер работы
\newcommand{\subject}{Базы данных}         % учебный предмет   
\newcommand{\variant}{№ 1909}                % номер варианта работы
\newcommand{\student}{Шубин Егор Вячеславович}    % определение ФИО студента
\newcommand{\studygroup}{P3109}                 % определение учебной группы 
\newcommand{\teacher}{% принимающий
    Лектор: Николаев Владимир Вячеславович,\\[1mm]% ФИО лектора
    Практик: Воронина Дарья Сергеевна % ФИО практика
}

\include{preamble}
\begin{document}
\def\contentsname{Содержание} 


\tableofcontents

\setcounter{page}{2}

\newpage
\Chapter{\lab\ \labnumber}{}{}
\Section{Задание варианта \variant}


Для отношений, полученных при построении предметной области из лабораторной работы №1, выполните следующие действия:

\begin{itemize}

\item Опишите функциональные зависимости для отношений полученной схемы (минимальное множество);

\item Приведите отношения в 3NF (как минимум). Постройте схему на основеNF (как минимум). 

\item Опишите изменения в функциональных зависимостях, произошедшие после преобразования в 3NF (как минимум). Постройте схему на основеNF;
Преобразуйте отношения в BCNF. Докажите, что полученные отношения представлены в BCNF. Если ваша схема находится уже в BCNF, докажите это;

\item Какие денормализации будут полезны для вашей схемы? Приведите подробное описание.

\end{itemize}
Придумайте триггер и связанную с ним функцию, относящиеся к вашей предметной области, согласуйте их с преподавателем и реализуйте на языке PL/pgSQL.
\newpage
\Section{Выполнение задания:}
\Subsection{Инфологическая модель:}

\Subsection{Опиcание функциональных зависимостей:}
\begin{itemize}
    \item ActionScene: \(id \rightarrow (PlaceId, ObjectId, ActionId)\)
    \item Place: \(id \rightarrow PlaceName\)
    \item Objects: \(id \rightarrow ObjectType, Characteristic\)
    \item Actions: \(id \rightarrow (ActionName, DirectionId, Purpose)\)
    \item ActionDescription: \(id \rightarrow Description\)
    \item ActionToType: \((ActionId, ActionDescribeId) \rightarrow 
 \varnothing\)
    \item Animal: \(id \rightarrow (ObjectId, Nickname, Age, AnimalSize)\)
    \item Human: \(id \rightarrow (ObjectId, HumanName, Age, Gender)\)
\end{itemize}
\Subsection{Нормализация:}
\begin{itemize}
    \item 1NF: Отношения находятся в 1НФ, т.к. все атрибуты являются простыми. Все домены содержат только скалярные значения, повторяющихся доменов в таблице нет.
    \item 2NF: Отношение находится во 2НФ, т.к. каждый не ключевой атрибут неприводимо зависит от Первичного Ключа(ПК).
    \item 3NF: В сущности Objects атрибут ObjectType может относится к разным кортежам в таблице, например значение атрибута "Человек" может относится к нескольким объектам, нужно вынести ObjectType в отдельную сущность.
    \item BCNF: Отношение удовлетворяет данной форме, т.к. у сущностей нет составных ключей, у каждой сущности единственный суперключ.
\end{itemize}
\Subsection{Нормализованная модель:}


\newpage
\Subsection{Денормализация}
Сущности можно денормализовать путем упрощения и соединения нескольких сущностей для того чтобы упростить запросы и увеличить производительность за счет уменьшения JOIN-запросов. Например можно не выносить ObjectType или ActionDescription в отдельную сущность.

\newpage
\Subsection{Функция и триггер на языке PL/pgSQL}

\begin{lstlisting}

\end{lstlisting}


\Section{Вывод}
В ходе лабораторной работы была проведена нормализация схемы базы данных до 3NF и BCNF, что устранило избыточность данных и аномалии обновления. Разработаны триггеры на PL/pgSQL для автоматического расчета статистики времени действий, обеспечивающие целостность данных. Выполнена частичная денормализация для оптимизации частых запросов, что улучшило производительность без потери согласованности данных.
\end{document}
